% generated from JIRA project LVV
% using template at /usr/local/lib/python3.7/site-packages/docsteady/templates/dm-tpr.latex.jinja2.
% using docsteady version 1.2rc24
% Please do not edit -- update information in Jira instead

\documentclass[DM,lsstdraft,STR,toc]{lsstdoc}
\usepackage{geometry}
\usepackage{longtable,booktabs}
\usepackage{enumitem}
\usepackage{arydshln}
\usepackage{attachfile}
\usepackage{array}

\newcolumntype{L}[1]{>{\raggedright\let\newline\\\arraybackslash\hspace{0pt}}p{#1}}

\input meta.tex

\newcommand{\attachmentsUrl}{https://github.com/\gitorg/\lsstDocType-\lsstDocNum/blob/\gitref/attachments}
\providecommand{\tightlist}{
  \setlength{\itemsep}{0pt}\setlength{\parskip}{0pt}}

\setcounter{tocdepth}{4}

\begin{document}

\def\milestoneName{Science Pipelines Fall 2019 Release}
\def\milestoneId{LDM-503-11b}
\def\product{Science Pipelines SW}

\setDocCompact{true}

\title{LDM-503-11b: Science Pipelines Fall 2019 Release Test Plan and Report}
\setDocRef{\lsstDocType-\lsstDocNum}
\date{\vcsdate}
\author{ John Swinbank }

% Most recent last
\setDocChangeRecord{
\addtohist{}{2019-11-21}{First Draft}{John Swinbank}
}

\setDocCurator{John Swinbank}
\setDocUpstreamLocation{\url{https://github.com/lsst-dm/\lsstDocType-\lsstDocNum}}
\setDocUpstreamVersion{\vcsrevision}



\setDocAbstract{
This is the test plan and report for
\textbf{ Science Pipelines Fall 2019 Release} (LDM-503-11b),
an LSST milestone pertaining to the Data Management Subsystem.
}


\maketitle

\section{Introduction}
\label{sect:intro}


\subsection{Objectives}
\label{sect:objectives}

 This test plan checks for the successful release of the Fall 2019
release of the LSST Science Pipelines (Pipelines release version
19.0.0).\\[2\baselineskip]It will demonstrate that:\\[2\baselineskip]

\begin{itemize}
\tightlist
\item
  The release has been tagged, built and made available through standard
  distribution channels;
\item
  Release documentation, including release notes and a characterization
  report, are available on the LSST Pipelines documentation website
  (https://pipelines.lsst.io/);
\item
  An end-user can follow standard instructions to install the release
  onto some representative system;
\item
  The release is installed into the ``shared stack'' on the lsst-dev
  shared developer systems and the Verification Cluster at the LSST Data
  Facility;
\item
  The lsst\_dm\_stack\_demo test package executes successfully in the
  context of the release.
\end{itemize}

This test plan does not, in itself, verify the scientific integrity or
algorithmic correctness of the release, beyond checking that defined
procedures for checking basic correctness and characterizing algorithmic
performance have been followed.\\[2\baselineskip]

\subsection{Scope}\label{scope}

The overall strategy for testing and verification within LSST Data
Management is described in \citeds{LDM-503}.\\
This test plan specifically verifies successful completion of milestone
LDM-503-11b, which refers to the Fall 2019 release of the LSST Science
Pipelines.



\subsection{System Overview}
\label{sect:systemoverview}

 The LSST Science Pipelines comprise the scientific algorithms that will
be used to process LSST data, arranged into executable pipelines by
means of the LSST ``task'' framework. They also include execution
middleware that is common across execution environments (for example,
the ``Data Butler'' I/O abstraction is included, but schedulers or
workflow management for specific clusters is not), and ``camera
packages'', which adapt and configure the algorithms for use with
specific instrumentation.\\[2\baselineskip]

\subsection{\texorpdfstring{Applicable Documents\\
}{Applicable Documents }}\label{applicable-documents}

\citeds{LDM-503} Data Management Test Plan\\
\citeds{LDM-151} Data Management Science Pipelines Design\\
\citeds{LSE-61} Data Management System Requirements


\subsection{Document Overview}
\label{sect:docoverview}

This document was generated from Jira, obtaining the relevant information from the
\href{https://jira.lsstcorp.org/secure/Tests.jspa\#/testPlan/LVV-P62}{LVV-P62}
~Jira Test Plan and related Test Cycles (
\href{https://jira.lsstcorp.org/secure/Tests.jspa\#/testCycle/LVV-C113}{LVV-C113}
).

Section \ref{sect:intro} provides an overview of the test campaign, the system under test (\product{}),
the applicable documentation, and explains how this document is organized.
Section \ref{sect:testplan} provides additional information about the test plan, like for example the configuration
used for this test or related documentation.
Section \ref{sect:personnel} describes the necessary roles and lists the individuals assigned to them.

Section \ref{sect:overview} provides a summary of the test results, including an overview in Table \ref{table:summary},
an overall assessment statement and suggestions for possible improvements.
Section \ref{sect:detailedtestresults} provides detailed results for each step in each test case.

The current status of test plan \href{https://jira.lsstcorp.org/secure/Tests.jspa\#/testPlan/LVV-P62}{LVV-P62} in Jira is \textbf{ Approved }.

\subsection{References}
\label{sect:references}
\renewcommand{\refname}{}
\bibliography{lsst,refs,books,refs_ads,local}


\newpage
\section{Test Plan Details}
\label{sect:testplan}


\subsection{Data Collection}

  Observing is not required for this test campaign.

\subsection{Verification Environment}
\label{sect:hwconf}
  Several of the tests described in this plan are agnostic of environment:
they involve checking that certain content has been properly published.
This can be performed from any internet-connected system with a web
browser, and will, in this case, likely be executed from the tester's
laptop.\\[2\baselineskip]Where tests require installation or execution
of specific Science Pipelines components, this will be carried out on
the ``lsst-dev'' shared developer infrastructure at the LSST Data
Facility. This infrastructure provides a number of powerful (high core
count, high RAM) systems accessible to LSST developers. At time of
writing, they are running CentOS 7.6.1810; in practice, any version of
CentOS (or a similar operating system) is appropriate for this test
plan, as long as it complies with the published installation
prerequisites of the LSST Science Pipelines.\\[2\baselineskip]




\subsection{Related Documentation}


No additional documentation provided.


\subsection{PMCS Activity}

Primavera milestones related to the test campaign:
\begin{itemize}
\item LDM-503-11b
\end{itemize}


\newpage
\section{Personnel}
\label{sect:personnel}

The personnel involved in the test campaign is shown in the following table.

{\small
\begin{longtable}{p{3cm}p{3cm}p{3cm}p{6cm}}
\hline
\multicolumn{2}{r}{T. Plan \href{https://jira.lsstcorp.org/secure/Tests.jspa\#/testPlan/LVV-P62}{LVV-P62} owner:} &
\multicolumn{2}{l}{\textbf{ John Swinbank } }\\\hline
\multicolumn{2}{r}{T. Cycle \href{https://jira.lsstcorp.org/secure/Tests.jspa\#/testCycle/LVV-C113}{LVV-C113} owner:} &
\multicolumn{2}{l}{\textbf{
John Swinbank }
} \\\hline
\textbf{Test Cases} & \textbf{Assigned to} & \textbf{Executed by} & \textbf{Additional Test Personnel} \\ \hline
\href{https://jira.lsstcorp.org/secure/Tests.jspa#/testCase/LVV-T362}{LVV-T362}
& {\small John Swinbank } & {\small John Swinbank } &
\begin{minipage}[]{6cm}
\smallskip
{\small  }
\medskip
\end{minipage}
\\ \hline
\href{https://jira.lsstcorp.org/secure/Tests.jspa#/testCase/LVV-T1601}{LVV-T1601}
& {\small John Swinbank } & {\small John Swinbank } &
\begin{minipage}[]{6cm}
\smallskip
{\small  }
\medskip
\end{minipage}
\\ \hline
\href{https://jira.lsstcorp.org/secure/Tests.jspa#/testCase/LVV-T363}{LVV-T363}
& {\small John Swinbank } & {\small John Swinbank } &
\begin{minipage}[]{6cm}
\smallskip
{\small  }
\medskip
\end{minipage}
\\ \hline
\end{longtable}
}

\newpage

\section{Test Campaign Overview}
\label{sect:overview}

\subsection{Summary}
\label{sect:summarytable}

{\small
\begin{longtable}{p{2cm}cp{2.3cm}p{8.6cm}p{2.3cm}}
\toprule
\multicolumn{2}{r}{ T. Plan \href{https://jira.lsstcorp.org/secure/Tests.jspa\#/testPlan/LVV-P62}{LVV-P62}:} &
\multicolumn{2}{p{10.9cm}}{\textbf{ LDM-503-11b: Science Pipelines Fall 2019 Release }} & Approved \\\hline
\multicolumn{2}{r}{ T. Cycle \href{https://jira.lsstcorp.org/secure/Tests.jspa\#/testCycle/LVV-C113}{LVV-C113}:} &
\multicolumn{2}{p{10.9cm}}{\textbf{ LDM-503-11b: Science Pipelines Fall 2019 Release }} & Done \\\hline
\textbf{Test Cases} &  \textbf{Ver.} & \textbf{Status} & \textbf{Comment} & \textbf{Issues} \\\toprule
\href{https://jira.lsstcorp.org/secure/Tests.jspa#/testCase/LVV-T362}{LVV-T362}
&  1
& Pass w/ Deviation &
\begin{minipage}[]{9cm}
\smallskip
Test completed with deviation; see linked issue.
\medskip
\end{minipage}
&   \href{https://jira.lsstcorp.org/browse/DM-26471}{DM-26471}
\\\hline
\href{https://jira.lsstcorp.org/secure/Tests.jspa#/testCase/LVV-T1601}{LVV-T1601}
&  1
& Pass &
\begin{minipage}[]{9cm}
\smallskip
Test completed successfully.
\medskip
\end{minipage}
&   \\\hline
\href{https://jira.lsstcorp.org/secure/Tests.jspa#/testCase/LVV-T363}{LVV-T363}
&  1
& Pass &
\begin{minipage}[]{9cm}
\smallskip
Test completed successfully.
\medskip
\end{minipage}
&   \\\hline
\caption{Test Campaign Summary}
\label{table:summary}
\end{longtable}
}

\subsection{Overall Assessment}
\label{sect:overallassessment}

The bulk of the tests described in this plan were completed
successfully, with almost no problems encountered.\\[2\baselineskip]The
one issue seen was in the documentation in the lsst\_dm\_stack\_demo
package not properly matching the on-disk files: documentation refers to
a bin/compare.py file which is executed using Python, whereas in fact
the package provides a bin/compare file which must be executed using
bash.\\[2\baselineskip]Note, however, that this package is obsolete and
will be removed from future releases (issue DM-22578). This is therefore
a minor issue. Nevertheless, issue DM-26471 has been filed to resolve
this problem.

\subsection{Recommended Improvements}
\label{sect:recommendations}

The lsst\_dm\_stack\_demo documentation its documentation should be
updated to reflect the available functionality.

\newpage
\section{Detailed Test Results}
\label{sect:detailedtestresults}

\subsection{Test Cycle LVV-C113 }

Open test cycle {\it \href{https://jira.lsstcorp.org/secure/Tests.jspa#/testrun/LVV-C113}{LDM-503-11b: Science Pipelines Fall 2019 Release}} in Jira.

Test Cycle name: LDM-503-11b: Science Pipelines Fall 2019 Release\\
Status: Done

This test cycle describes tests performed on the Science Pipelines Fall
2019 (v19.0.0) release, ensuring that the release is properly
identified, documented, distributed, installable and tested.

\subsubsection{Software Version/Baseline}
A web browser is required for inspecting release artifacts published to
the web.\\[2\baselineskip]Testing the software installation procedures,
and demonstrating that the release's integration tests can be executed
successfully, require a supported operating system with the documented
prerequisites of the release installed. This will be carried out on the
''lsst-dev'' shared developer systems at the LSST Data
Facility.\\[2\baselineskip]At time of writing, Data Facility systems run
CentOS Linux release 7.6.1810, and it is anticipated that this will be a
supported platform for the Science Pipelines
release.\\[2\baselineskip]Science Pipelines prerequisites are documented
at \href{https://pipelines.lsst.io/}{pipelines.lsst.io}; all of these
must be installed.\\[2\baselineskip]It is possible that the software
release will involve a reorganization of documentation describing
prerequisites; in this case, the documentation corresponding to the new
release must be consulted.

\subsubsection{Configuration}
No specific configuration is required beyond the availability of test
systems with the prerequisite software, described above, installed.

\subsubsection{Test Cases in LVV-C113 Test Cycle}

\paragraph{ LVV-T362 - Installation of the LSST Science Pipelines Payloads }\mbox{}\\

Version \textbf{1}.
Open  \href{https://jira.lsstcorp.org/secure/Tests.jspa#/testCase/LVV-T362}{\textit{ LVV-T362 } }
test case in Jira.

This test will check that:

\begin{itemize}
\tightlist
\item
  The Alert Production Pipeline payload is available for installation
  from documented channels;
\item
  The Data Release Production Pipeline payload is available for
  installation from documented channels;
\item
  The Calibration Products Production Pipeline payload is available for
  installation from documented channels;
\item
  These payloads can be installed on systems at the LSST Data Facility
  following available documentation;
\item
  The installed pipeline payloads are capable of successfully executing
  basic integration tests.
\end{itemize}

Note that this test assumes a 2018-era packaging of the Science
Pipelines software, in which all the above payloads are represented by a
single ``meta-package'', lsst\_distrib.

\textbf{ Preconditions}:\\


Execution status: {\bf Pass w/ Deviation }

Final comment:\\Test completed with deviation; see linked issue.

Issues found during the execution of LVV-T362 test case:

\begin{itemize}
\item \href{https://jira.lsstcorp.org/browse/DM-26471}{DM-26471}~~README for lsst\_dm\_stack\_demo is incorrect

\end{itemize}

Detailed steps results:

\begin{longtable}{p{1cm}p{15cm}}
\hline
{Step} & Step Details\\ \hline
1 & Description \\
 & \begin{minipage}[t]{15cm}
{\footnotesize
The LSST Science Pipelines, described by the lsst\_distrib meta-package,
should be installed following the documentation available at
https://pipelines.lsst.io/. The suggested Conda environment will be used
to ensure that a supported execution environment is available.

\medskip }
\end{minipage}
\\ \cdashline{2-2}


 & Expected Result \\
 & \begin{minipage}[t]{15cm}{\footnotesize
Detailed output will depend on the installation method chosen, but will
confirm the successful installation of the Science Pipelines.

\medskip }
\end{minipage} \\ \cdashline{2-2}

 & Actual Result \\
 & \begin{minipage}[t]{15cm}{\footnotesize
The Science Pipelines were installed using the Docker image
7-stack-lsst\_distrib-v19\_0\_0, using the procedure described at
https://pipelines.lsst.io/install/docker.html\#quick-start. The command
executed was:\\[2\baselineskip]docker run -ti
docker.io/lsstsqre/centos:7-stack-lsst\_distrib-v19\_0\_0\\[2\baselineskip]After
a few moments, a shel inside the Docker container was available.

\medskip }
\end{minipage} \\ \cdashline{2-2}

 & Status: \textbf{ Pass } \\ \hline

2 & Description \\
 & \begin{minipage}[t]{15cm}
{\footnotesize
The lsst\_distrib top-level metapackage will be enabled. Assuming that
the software has been installed at
\$\{LSST\_DIR\}:\\[2\baselineskip]\hspace*{0.333em} ~ ~ ~source
\$\{LSST\_DIR\}/loadLSST.bash\\
\hspace*{0.333em} ~ ~ ~setup lsst\_distrib

\medskip }
\end{minipage}
\\ \cdashline{2-2}


 & Expected Result \\
 & \begin{minipage}[t]{15cm}{\footnotesize
Nothing is printed. The command\\[2\baselineskip]\hspace*{0.333em} ~eups
list -s lsst\_distrib\\[2\baselineskip]may be used to confirm that the
correct version of the codebase has been installed.

\medskip }
\end{minipage} \\ \cdashline{2-2}

 & Actual Result \\
 & \begin{minipage}[t]{15cm}{\footnotesize
{[}lsst@cdc991ebdb1d stack{]}\$ source
/opt/lsst/software/stack/loadLSST.bash\\
(lsst-scipipe-4d7b902) {[}lsst@cdc991ebdb1d stack{]}\$ setup
lsst\_distrib\\
(lsst-scipipe-4d7b902) {[}lsst@cdc991ebdb1d stack{]}\$ eups list -s
lsst\_distrib\\
19.0.0+2 current v19\_0\_0 setup

\medskip }
\end{minipage} \\ \cdashline{2-2}

 & Status: \textbf{ Pass } \\ \hline

3 & Description \\
 & \begin{minipage}[t]{15cm}
{\footnotesize
The ``LSST Stack Demo'' package will be downloaded onto the test system
from https://github.com/lsst/lsst\_dm\_stack\_demo/releases. The version
corresponding to to the version of the Science Pipelines under test
should be chosen.

\medskip }
\end{minipage}
\\ \cdashline{2-2}


 & Expected Result \\
 & \begin{minipage}[t]{15cm}{\footnotesize
Depends on the tool selected by the user for downloading.

\medskip }
\end{minipage} \\ \cdashline{2-2}

 & Actual Result \\
 & \begin{minipage}[t]{15cm}{\footnotesize
(lsst-scipipe-4d7b902) {[}lsst@cdc991ebdb1d \textasciitilde{}{]}\$ wget
https://github.com/lsst/lsst\_dm\_stack\_demo/archive/19.0.0.tar.gz\\
--2020-08-25 21:40:27--
https://github.com/lsst/lsst\_dm\_stack\_demo/archive/19.0.0.tar.gz\\
Resolving github.com (github.com)\ldots{} 192.30.255.112\\
Connecting to github.com
(github.com)\textbar{}192.30.255.112\textbar{}:443\ldots{} connected.\\
HTTP request sent, awaiting response\ldots{} 302 Found\\
Location:
https://codeload.github.com/lsst/lsst\_dm\_stack\_demo/tar.gz/19.0.0
{[}following{]}\\
--2020-08-25 21:40:27--
https://codeload.github.com/lsst/lsst\_dm\_stack\_demo/tar.gz/19.0.0\\
Resolving codeload.github.com (codeload.github.com)\ldots{}
192.30.255.121\\
Connecting to codeload.github.com
(codeload.github.com)\textbar{}192.30.255.121\textbar{}:443\ldots{}
connected.\\
HTTP request sent, awaiting response\ldots{} 200 OK\\
Length: unspecified {[}application/x-gzip{]}\\
Saving to: `19.0.0.tar.gz'\\[2\baselineskip]{[}
\textless{}=\textgreater{} {]} 34,556,522 6.23MB/s in
6.5s\\[2\baselineskip]2020-08-25 21:40:34 (5.07 MB/s) - `19.0.0.tar.gz'
saved {[}34556522{]}\\[2\baselineskip]\hspace*{0.333em}

\medskip }
\end{minipage} \\ \cdashline{2-2}

 & Status: \textbf{ Pass } \\ \hline

4 & Description \\
 & \begin{minipage}[t]{15cm}
{\footnotesize
The stack demo package is uncompressed into a directory \$\{DEMO\_DIR\}.

\medskip }
\end{minipage}
\\ \cdashline{2-2}


 & Expected Result \\
 & \begin{minipage}[t]{15cm}{\footnotesize
Depends on options given to the tar command. Should confirm the
availability of the stack demo source.

\medskip }
\end{minipage} \\ \cdashline{2-2}

 & Actual Result \\
 & \begin{minipage}[t]{15cm}{\footnotesize
(lsst-scipipe-4d7b902) {[}lsst@cdc991ebdb1d \textasciitilde{}{]}\$ tar
zxf 19.0.0.tar.gz\\
(lsst-scipipe-4d7b902) {[}lsst@cdc991ebdb1d \textasciitilde{}{]}\$

\medskip }
\end{minipage} \\ \cdashline{2-2}

 & Status: \textbf{ Pass } \\ \hline

5 & Description \\
 & \begin{minipage}[t]{15cm}
{\footnotesize
The demo package will be executed by following the instructions in its
README file.~

\medskip }
\end{minipage}
\\ \cdashline{2-2}


 & Expected Result \\
 & \begin{minipage}[t]{15cm}{\footnotesize
Successful execution will result in the string ``Ok'' being returned.

\medskip }
\end{minipage} \\ \cdashline{2-2}

 & Actual Result \\
 & \begin{minipage}[t]{15cm}{\footnotesize
(lsst-scipipe-4d7b902) {[}lsst@cdc991ebdb1d
lsst\_dm\_stack\_demo-19.0.0{]}\$ setup -r .\\
(lsst-scipipe-4d7b902) {[}lsst@cdc991ebdb1d
lsst\_dm\_stack\_demo-19.0.0{]}\$ ./bin/demo.sh\\[2\baselineskip](many
lines of output, elided for
clarity)\\[2\baselineskip]processCcd.calibrate.photoCal INFO: Magnitude
zero point: 26.068660 +/- 0.001258 from 63 stars\\
processCcd.calibrate INFO: Photometric zero-point: 26.068660\\
Loading exposure registry from
/home/lsst/lsst\_dm\_stack\_demo-19.0.0/input/registry.sqlite3\\[2\baselineskip]Processing
completed successfully. The results are in
detected-sources.txt.\\[2\baselineskip](lsst-scipipe-4d7b902)
{[}lsst@cdc991ebdb1d lsst\_dm\_stack\_demo-19.0.0{]}\$ python
bin/compare.py detected-sources.txt\\
python: can't open file `bin/compare.py': {[}Errno 2{]} No such file or
directory\\[2\baselineskip](lsst-scipipe-4d7b902) {[}lsst@cdc991ebdb1d
lsst\_dm\_stack\_demo-19.0.0{]}\$ ./bin/compare detected-sources.txt\\
Ok.\\[2\baselineskip]Note that the documented bin/compare.py command was
NOT functional, but an obvious equivalent was available.

\medskip }
\end{minipage} \\ \cdashline{2-2}

 & Status: \textbf{ Pass w/ Deviation } \\ \hline

\end{longtable}

\paragraph{ LVV-T1601 - Science Pipelines available on developer hardware }\mbox{}\\

Version \textbf{1}.
Open  \href{https://jira.lsstcorp.org/secure/Tests.jspa#/testCase/LVV-T1601}{\textit{ LVV-T1601 } }
test case in Jira.

This test will check that a given release of the LSST Science Pipelines
is available for use in a ``shared stack'' on developer infrastructure.

\textbf{ Preconditions}:\\


Execution status: {\bf Pass }

Final comment:\\Test completed successfully.


Detailed steps results:

\begin{longtable}{p{1cm}p{15cm}}
\hline
{Step} & Step Details\\ \hline
1 & Description \\
 & \begin{minipage}[t]{15cm}
{\footnotesize
Consult the LSST Developer Guide (http://developer.lsst.io/) to
establish:\\[2\baselineskip]

\begin{itemize}
\tightlist
\item
  An appropriate hostname and login instructions for the shared
  developer infrastructure at the LSST Data Facility;
\item
  Instructions for initializing the shared stack on the developer
  infrastructure.
\end{itemize}

\medskip }
\end{minipage}
\\ \cdashline{2-2}


 & Expected Result \\
 & \begin{minipage}[t]{15cm}{\footnotesize
The Developer Guide clearly presents information about connecting to and
using shared infrastructure.

\medskip }
\end{minipage} \\ \cdashline{2-2}

 & Actual Result \\
 & \begin{minipage}[t]{15cm}{\footnotesize
\begin{itemize}
\tightlist
\item
  Instructions for connecting to shared hardware are available at
  https://developer.lsst.io/services/lsst-login.html
\item
  Instructions for initializing the shared stack, along with a range of
  other software, are available at
  https://developer.lsst.io/services/software.html
\end{itemize}

\medskip }
\end{minipage} \\ \cdashline{2-2}

 & Status: \textbf{ Pass } \\ \hline

2 & Description \\
 & \begin{minipage}[t]{15cm}
{\footnotesize
Connect to the shared infrastructure following the Developer Guide
instructions.

\medskip }
\end{minipage}
\\ \cdashline{2-2}


 & Expected Result \\
 & \begin{minipage}[t]{15cm}{\footnotesize
A shell prompt on a shared machine.

\medskip }
\end{minipage} \\ \cdashline{2-2}

 & Actual Result \\
 & \begin{minipage}[t]{15cm}{\footnotesize
Login successful. A prompt is available.

\medskip }
\end{minipage} \\ \cdashline{2-2}

 & Status: \textbf{ Pass } \\ \hline

3 & Description \\
 & \begin{minipage}[t]{15cm}
{\footnotesize
Initialize the LSST environment following the Developer Guide
instructions.

\medskip }
\end{minipage}
\\ \cdashline{2-2}

 & Example Code \\
 & \begin{minipage}[t]{15cm}{\footnotesize
source /software/lsstsw/stack/loadLSST.bash

\medskip }
\end{minipage} \\ \cdashline{2-2}

 & Expected Result \\
 & \begin{minipage}[t]{15cm}{\footnotesize
No errors are shown.

\medskip }
\end{minipage} \\ \cdashline{2-2}

 & Actual Result \\
 & \begin{minipage}[t]{15cm}{\footnotesize
The shared stack for the 19.0.0 release is available in
/software/lsstsw/stack\_20191101. The command executed
was:\\[2\baselineskip]source
software/lsstsw/stack\_20191101/loadLSST.bash

\medskip }
\end{minipage} \\ \cdashline{2-2}

 & Status: \textbf{ Pass } \\ \hline

4 & Description \\
 & \begin{minipage}[t]{15cm}
{\footnotesize
List available software products using EUPS, and check that the release
under test is available.

\medskip }
\end{minipage}
\\ \cdashline{2-2}

 & Example Code \\
 & \begin{minipage}[t]{15cm}{\footnotesize
eups list lsst\_distrib \textbar{} grep vXX\_XX\_XX ~\# replace
XX\_XX\_XX with the version number of the release under test

\medskip }
\end{minipage} \\ \cdashline{2-2}

 & Expected Result \\
 & \begin{minipage}[t]{15cm}{\footnotesize
The provided version number should be available in the shared stack.

\medskip }
\end{minipage} \\ \cdashline{2-2}

 & Actual Result \\
 & \begin{minipage}[t]{15cm}{\footnotesize
{[}swinbank@lsst-login01 \textasciitilde{}{]}\$ eups list lsst\_distrib
\textbar{} grep 19\_0\_0\\
19.0.0+2 current v19\_0\_0

\medskip }
\end{minipage} \\ \cdashline{2-2}

 & Status: \textbf{ Pass } \\ \hline

\end{longtable}

\paragraph{ LVV-T363 - Science Pipelines Release Documentation }\mbox{}\\

Version \textbf{1}.
Open  \href{https://jira.lsstcorp.org/secure/Tests.jspa#/testCase/LVV-T363}{\textit{ LVV-T363 } }
test case in Jira.

This test will check:

\begin{itemize}
\tightlist
\item
  That a particular Science Pipelines release is adequately described by
  documentation at the https://pipelines.lsst.io/ site;
\item
  That the Science Pipelines release is accompanied by a
  characterization report which describes its scientific performance.
\end{itemize}

\textbf{ Preconditions}:\\


Execution status: {\bf Pass }

Final comment:\\Test completed successfully.


Detailed steps results:

\begin{longtable}{p{1cm}p{15cm}}
\hline
{Step} & Step Details\\ \hline
1 & Description \\
 & \begin{minipage}[t]{15cm}
{\footnotesize
Load the Science Pipelines website at https://pipelines.lsst.io/.

\medskip }
\end{minipage}
\\ \cdashline{2-2}


 & Expected Result \\
 & \begin{minipage}[t]{15cm}{\footnotesize
The website is displayed.

\medskip }
\end{minipage} \\ \cdashline{2-2}

 & Actual Result \\
 & \begin{minipage}[t]{15cm}{\footnotesize
Website is displayed.

\medskip }
\end{minipage} \\ \cdashline{2-2}

 & Status: \textbf{ Pass } \\ \hline

2 & Description \\
 & \begin{minipage}[t]{15cm}
{\footnotesize
Identify documentation for the release under test. This should be
clearly labelled on the documentation site.\\[2\baselineskip]If the
latest release is being tested, the default page loaded when visiting
https://pipelines.lsst.io/ should be the documentation
required.\\[2\baselineskip]If this test is for another release, the site
should present clear instructions for changing the edition (or version)
of the documentation being examined, and documentation for the release
under test should be available.

\medskip }
\end{minipage}
\\ \cdashline{2-2}


 & Expected Result \\
 & \begin{minipage}[t]{15cm}{\footnotesize
The documentation for the release under test is displayed.

\medskip }
\end{minipage} \\ \cdashline{2-2}

 & Actual Result \\
 & \begin{minipage}[t]{15cm}{\footnotesize
Documentation for the release under test, v19\_0\_0, is clearly
displayed at https://pipelines.lsst.io/v/v19\_0\_0/index.html.

\medskip }
\end{minipage} \\ \cdashline{2-2}

 & Status: \textbf{ Pass } \\ \hline

3 & Description \\
 & \begin{minipage}[t]{15cm}
{\footnotesize
Inspect the documentation to ensure that it refers to the release under
test, and that it provides:

\begin{itemize}
\tightlist
\item
  Release notes, describing changes in this release relative to the
  previous;
\item
  Installation instructions, together with a list of supported platforms
  and prerequisites;
\item
  Getting started information.
\end{itemize}

\medskip }
\end{minipage}
\\ \cdashline{2-2}


 & Expected Result \\
 & \begin{minipage}[t]{15cm}{\footnotesize
The user is satisfied that the required information is available.

\medskip }
\end{minipage} \\ \cdashline{2-2}

 & Actual Result \\
 & \begin{minipage}[t]{15cm}{\footnotesize
\begin{itemize}
\tightlist
\item
  Release notes:
  https://pipelines.lsst.io/v/v19\_0\_0/releases/notes.html\#release-v19-0-0
\item
  Installation instructions:
  https://pipelines.lsst.io/v/v19\_0\_0/index.html\#installation
\item
  Getting started information:
  https://pipelines.lsst.io/v/v19\_0\_0/index.html\#getting-started
\end{itemize}

\medskip }
\end{minipage} \\ \cdashline{2-2}

 & Status: \textbf{ Pass } \\ \hline

4 & Description \\
 & \begin{minipage}[t]{15cm}
{\footnotesize
Locate the Characterization Metric Report corresponding to this release.
It should be linked from the main release documentation.

\medskip }
\end{minipage}
\\ \cdashline{2-2}


 & Expected Result \\
 & \begin{minipage}[t]{15cm}{\footnotesize
The user is satisfied that the report is available.

\medskip }
\end{minipage} \\ \cdashline{2-2}

 & Actual Result \\
 & \begin{minipage}[t]{15cm}{\footnotesize
Characterization metric report for this release is DMTR-191. It is
clearly linked from
https://pipelines.lsst.io/v/v19\_0\_0/metrics.html\#characterization-metric-reports.

\medskip }
\end{minipage} \\ \cdashline{2-2}

 & Status: \textbf{ Pass } \\ \hline

5 & Description \\
 & \begin{minipage}[t]{15cm}
{\footnotesize
Verify that the characterization metric report describes the scientific
performance of the release in terms of a selection of performance
metrics drawn from high-level requirements documentation (the Science
Requirements Document, LPM-17; the LSST System Requirements, LSE-29;
and/or the Observatory System Specifications, LSE-30).

\medskip }
\end{minipage}
\\ \cdashline{2-2}


 & Expected Result \\
 & \begin{minipage}[t]{15cm}{\footnotesize
Metric values describing the performance of the release, for example as
computed by validate\_drp, are described in the report.

\medskip }
\end{minipage} \\ \cdashline{2-2}

 & Actual Result \\
 & \begin{minipage}[t]{15cm}{\footnotesize
The characterization metric report has been inspected and seen to
contain appropriate metrics.

\medskip }
\end{minipage} \\ \cdashline{2-2}

 & Status: \textbf{ Pass } \\ \hline

\end{longtable}


\input{appendix.tex}
\end{document}
